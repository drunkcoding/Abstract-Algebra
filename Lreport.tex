\input project
\usepackage{natbib}
\usepackage{multirow}
%\usepackage{tipa}
%\usepackage[tone]{tipa}
%\usepackage{CJK}
\usepackage{indentfirst}
%\pagestyle{headings}
%\pagestyle{fancy}
%\fancyfoot[C]{UM\_SJTU JI VE401 Project Report Page \thepage}
\renewcommand\thesection{\arabic {section}}
\setlength{\parindent}{2em}
%\numberwithin{equation}{subsection}

\title{Abstract Algebra Summary}
\author{Xue Leyang}

\begin{document}
	\maketitle
	%\tableofcontents
	\setcounter{page}{0} 
	\thispagestyle{empty}
	%\addcontentsline{toc}{section}{Appendix}
	%\newpage
	\section{The System of Natural Number}
	\subsection{Prosuct Set}
		\begin{definition}
		The product set $S \times T$ ofteo arbitrary sets $S$ and $T$ is a set of pairs $(s,t), s\in S, t\in T$.\\
		In general $\prod S_i = S_1\times S_2\times\cdots\times S_r$ is the collection of r-tuples $(s_1, s_2, \cdots, s_r)$, where $s_i\in S_i$.\\
		If $(s_1, s_2, \cdots, s_r)$ and $(s_1', s_2', \cdots, s_r')$ are equal, we have $s_1=s_1',s_2=s_2', \cdots, s_r=s_r'$.
		\end{definition}
	\subsection{Mapping}
		\begin{definition}
		A mapping $\a$ of set $S$ onto set $T$ if $\forall t\in T, \exists s\in S \implies \a(s)=t$, we also write the image of $s$ in $T$ as $s\a$ or $s^\a$. The image set of $S$ is denoted as $S\a$ or $S^\a$.\\
		If $\a$ is a one-to-one mapping, $s$ is unique for every $t$, and we call a inverse mapping $\a^{-1}$ which is one-to-one of $T$ onto $S$
		\end{definition}
		\begin{definition}
		Resultant or Product of mapping is denoted as $\a\b$, where $\a$ maps set $S$ onto set $T$ and $\b$ maps set $T$ onto set $U$. Mapping of $S$ onto $U$ can be written as $S(\a\b)=(S\a)\b$, and same for each element
		\end{definition}
		\begin{theorem}
		content...
		\end{theorem}
	
	\section{Groups}
	\section{Rings}
	\section{Fields}
	
\end{document}