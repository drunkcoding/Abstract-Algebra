\input project
\usepackage{natbib}
\usepackage{multirow}
%\usepackage{tipa}
%\usepackage[tone]{tipa}
%\usepackage{CJK}
\usepackage{indentfirst}
%\pagestyle{headings}
%\pagestyle{fancy}
%\fancyfoot[C]{UM\_SJTU JI VE401 Project Report Page \thepage}
\renewcommand\thesection{\arabic {section}}
\setlength{\parindent}{2em}
%\numberwithin{equation}{subsection}

\title{Abstract Algebra Summary}
\author{Xue Leyang}

\begin{document}
	\maketitle
	%\tableofcontents
	\setcounter{page}{0} 
	\thispagestyle{empty}
	%\addcontentsline{toc}{section}{Appendix}
	%\newpage
	\section{The System of Natural Number}
	\subsection{Prosuct Set}
		\begin{definition}
		The product set $S \times T$ ofteo arbitrary sets $S$ and $T$ is a set of pairs $(s,t), s\in S, t\in T$.\\
		In general $\prod S_i = S_1\times S_2\times\cdots\times S_r$ is the collection of r-tuples $(s_1, s_2, \cdots, s_r)$, where $s_i\in S_i$.\\
		If $(s_1, s_2, \cdots, s_r)$ and $(s_1', s_2', \cdots, s_r')$ are equal, we have $s_1=s_1',s_2=s_2', \cdots, s_r=s_r'$.
		\end{definition}
	\subsection{Mapping}
		\begin{definition}
		A mapping $\a$ of set $S$ onto set $T$ if $\forall t\in T, \exists s\in S \implies \a(s)=t$, we also write the image of $s$ in $T$ as $s\a$ or $s^\a$. The image set of $S$ is denoted as $S\a$ or $S^\a$.\\
		If $\a$ is a one-to-one mapping, $s$ is unique for every $t$, and we call a inverse mapping $\a^{-1}$ which is one-to-one of $T$ onto $S$
		\end{definition}
		\begin{definition}
		Resultant or Product of mapping is denoted as $\a\b$, where $\a$ maps set $S$ onto set $T$ and $\b$ maps set $T$ onto set $U$. Mapping of $S$ onto $U$ can be written as $S(\a\b)=(S\a)\b$, and same for each element.
		\end{definition}
		\begin{remark}
		Mapping of a set into itself is called \textit{transformations} of sets, including Identity mapping that leave all element in $S$ fixed.
		\end{remark}
		\begin{definition}
		Identity mapping, denoted $\id_S$, its product with any transformation $\a$, $\id_S\a = \a = \a\id_S$
		\end{definition}
		\begin{theorem}
		If $\a$ is a one-to-one mapping of $S$ onto $T$ and has inverse $\a^{-1}$, then $\a\a^{-1}=\id_S$ and $\a^{-1}\a=\id_T$. Conversely, if $\a:S\mapsto T$ and $\b:T \mapsto S$ such that $\a\b=\id_S$ and $\b\a=\id_T$, then $\a,\b$ are one-to-one mapping and $\b = \a^{-1}$
		\end{theorem}
		\begin{theorem}
		Associaltive law holds for the resultant of transformation of one set.
		\end{theorem}
		\begin{proof}
		Suppose we have for sets $S,T,U,V$ and $\a:S\mapsto T;\b:T\mapsto U;\gam:U\mapsto V$. Then $\forall x\in S$, $x((\a\b)\gam)=(x(\a\b))\gam=((x\a)\b)\gam$ and $x(\a(\b\gam))=(x\a)(\b\gam)=((x\a)\b)\gam$, hence $x((\a\b)\gam)=x(\a(\b\gam))$
		\end{proof}
	\subsection{Equivalence Relations}
		\begin{definition}\label{relations}
		The equivalence relation $\sim$ of a pair of element satisfies:
		\begin{enumerate}
		\item a $\sim$ a (reflecive property).
		\item a $\sim$ b \implies b $\sim$ a (symmetric property)
		\item a $\sim$ b and b $\sim$ c \implies a $\sim$ c (transitive property)
 		\end{enumerate}
		\end{definition}
		\begin{definition}
		We have a relation $\sim$ defined on a set $S$, an equivalence class is the subset of $S$ of element $b$ such that $b\sim a$.
		\end{definition}
		\begin{theorem}
		Two equivalence classes are either identical or mutually exclusive
		\end{theorem}
		\begin{proof}
		Suppose we have an equivalence class $[a]$ of element $a$, if $b\in [a]$, then $[b]\subseteq[a]$; hence by maximality of $[b]$, we conclude $[b]=[a]$.
		\end{proof}
		\begin{corollary}
		The collection of distinct equivalence classes gives a decomposition of the set $S$ into multually exculsive subsets. Conversely, suppose a set $S=\cup S_i$, where $S_i$ are multually exclusive, we can define relation $\mathcal{R}$ as $a\sim b \iff $ subsets $S_i,S_j$ containing $a,b$ are identical.
		\end{corollary}
		\begin{definition}
		The quotient set of $S$ with equivalence relation $\mathcal{R}$, denoted $S\backslash\mathcal{R}$, is the collection of all equivalence classes in $S$, where $s\mapsto[s]$ and $S\mapsto S\backslash\mathcal{R}$, i.e. each element maps to its equivalence class.
		\end{definition}
	\section{Semi-Groups and Groups}
	\subsection{Semi-Groups}
		\begin{definition}
		A semi-group is a system consisting of a set $\semiG$ and an associative binary composition in $\semiG$. i.e. $\forall a,b,c \in \semiG$, we have $(ab)c=a(bc)$.
		\end{definition}
		\begin{definition}
		Element $a$ and $b$ are said to be commute if $ab=ba, a,b\in\semiG$. If it holds for any pair $a,b$ in $\semiG$ then $\semiG$ is called commutative.
		\end{definition}
		\begin{definition}
		An element $e_l\in\semiG$ is called left identity if $\forall a\in\semiG, e_la=a$. Similarly, $e_r$ is right identity if $\forall a\in\semiG, e_ra=a$.
		\end{definition}
		\begin{theorem}
		If $e_l,e_r$ both exists in $\semiG$, then $e_l=e_r$, i.e. if two side identity exists then it is unique.
		\end{theorem}
		\begin{proof}
		We have $e_r=e_re_l=e_l$, from the definition of identity looking from two sides.
		\end{proof}
		\begin{definition}
		A right regular(unit) $a$ and right inverse $a'$, if $a, a'\in\semiG, aa'=e$, two side inverse $a^{-1}$
		\end{definition}
		\begin{theorem}
		If right inverse and left inverse both exists, they are identical. 
		\end{theorem}
		\begin{proof}
		We set $a,a',a''\in\semiG,$ that $aa'=e, a''a=e$, conclude that $a'=(a''a)a'=a''(aa')=a''$
		\end{proof}
	\subsection{Groups}
		\begin{definition}\label{defGroup}
		A group $\group$ is a semi-group that has an identity $e$ and in which every element is a unit.
		\begin{enumerate}
		\item associativity
		\item Exist $e\in\group$, $\forall a\in\group$ such that $ae=a=ea$
		\item $\forall a\in\group$ exist $a^{-1}$ such that $aa^{-1}=e=a^{-1}a$
		\end{enumerate}
		\end{definition}
		\begin{definition}
		$\forall a,b,c\in\semiG$, we have $ab=ac\implies b=c$ is called left cancellation and so is right cancellation that $\forall a,b,c\in\semiG$, we have $ba=ca\implies b=c$.
		\end{definition}
		\begin{theorem}\label{eqingroup}
		With $a,b\in\group$, the linear equation $ax=b$ the only solution $a^{-1}b$. Also $ya=b$ has solution $ab^{-1}$.
		\end{theorem}
		\begin{proof}
		If the solution is not unique, we set another solution to be $x'$, so that $ax=ax'$, contrasting to the result of left cancellation.
		\end{proof}
		\begin{theorem}
		The only idemponent ($\exists a\in\semiG, a^2=a$) in a group is the identity (unity).
		\end{theorem}
		\begin{proof}
		From $a\circ a=a$ we can observe $a$ hold both the property of left and right unit, the $a$ is a unit and unit is unique.
		\end{proof}
		\begin{theorem}
		A semi-group $\semiG$ with left unit $e_L$ and left inverse $a^{-1}_L, \forall a\in\semiG$, then it is a group. Also with right unit and right inverse.
		\end{theorem}
		\begin{proof}
		We take $\forall a\in\semiG$, $aa^{-1}_L=e_Laa^{-1}_L=\lr[]{(a^{-1}_L)^{-1}_La^{-1}_L}aa^{-1}_L=(a^{-1}_L)^{-1}_L(a^{-1}_La)a^{-1}_L=e_L$\\
		$ae_L=a(a^{-1}_La)=e_La=a$, so that we have equivalence property of right and left. 
		\end{proof}
		\begin{theorem}
		If $\semiG$ is a semi-group and linear equation $ya=b; ax=b, \forall a,b\in\semiG$ is solvable, then $\semiG$ is a group.
		\end{theorem}
		\begin{proof}
		We suppose $e$ is the solution for eqution $ya=a, \forall a\in\semiG$, and that the solution for $ax=b$ is $g$. Then $eb=e(ag)=ag=b$, $e$ is proved to be a left identity. Also, $yb=e$ is always solvable, so $\semiG$ has right inverse and right identity, thus is a group. 
		\end{proof}
		\begin{theorem}
		A finite semi-group with cancellation laws hold is a group.
		\end{theorem}
		\begin{proof}
		Let $\semiG=(a_1, a_2,\dots,a_n)$ has n distinct element, take $a,b\in\semiG$, and let $\mathfrak{T}=(aa_1, aa_2,\dots,aa_n)\implies \mathfrak{T}\subset\semiG$ according to the self-mapping of $\semiG$. $aa_i=aa_j\implies a_i=a_j$ by cancellation law, hence $\mathfrak{T}$ also have n distinct elements \implies $\mathfrak{T}=\semiG$. So that the linear equation $ax=b$ is solvable in $\semiG$, so as $ya=b$.
		\end{proof}
	\subsection{Subgroups}
		\begin{definition}
		If a set $\subG$ is a non-empty subset of (semi)group $\semiG$ and has property
		\begin{enumerate}
		\item closure i.e. $a,b\in\subG\implies ab\in\subG$
		\item Exist $e\in\subG$, $\forall a\in\subG$ such that $ae=a=ea$
		\item $\forall a\in\subG$ exist $a^{-1}$ such that $aa^{-1}=e=a^{-1}a$
		\end{enumerate}
		determines a sub-(semi)group of $\semiG$
		\end{definition}
		\begin{theorem}\label{sub-id-inv}
		Let $\subG$ to be the subgroup of $\group$, the identity in $\group$ is the identity in $\subG$, and $\forall a\in\subG$ the inverse in $\group$ is also the inverse in $\subG$.
		\end{theorem}
		\begin{theorem}
		A non-empty subset $\subG$ of a group $\group$ is a subgroup iff $\forall a,b\in\subG, ab^{-1}\in\subG$.
		\end{theorem}
		\begin{proof}
		If $\subG$ is a subgroup, then proved from Theorem \ref{eqingroup}. \\
		$\forall a\in\subG$, we have $e=aa^{-1}\in\subG$ and implies $a^{-1}=ea^{-1}\in\subG$, so that $\subG$ contains an element and its inverse, hence $ab=a(b^{-1})^{-1}\in\subG$ proves closure.
		\end{proof}
		\begin{theorem}
		If $A$ is collection of any subgroup $\subG$ of $\group$, then the intersection $\displaystyle\bigcap\limits_{A}\subG$ is a subgroup.
		\end{theorem}
		\begin{theorem}
		The centralizer $\centralizer(S), S\subset\group$ of $\group$ (the set of elements of $\group$ that commute with each element of $S$) is a subgroup of $\group$
		\end{theorem}
		\begin{proof}
		We take $a,b\in\centralizer(S), x\in S$, so that $(ab)x=a(bx)=a(xb)=(ax)b=(xa)b=x(ab)\implies ab\in\centralizer(S)$. And for $\forall a\in\centralizer(S)$, we have $ax=xa\implies axa^{-1}=x \implies xa^{-1}=a^{-1}x\implies a^{-1}\in\centralizer(S)$. The identity exists, so the $\centralizer(S)$ is a subgroup of $\group$.
		\end{proof}
	\subsection{Isomorphism}
		\begin{definition}
		Two groups $\group$ and $\group'$ are said to be isomorphic if there exists a 1-1 mapping $x\mapsto x'$ of $\group$(Isomorphism) onto $\group'$ such that $(xy)'=x'y'$. $\group$ and $\group'$ are said to be \textit{abstractly equivalent}, written as $\group\cong\group'$.
		\end{definition}
		\begin{theorem}
		Isomorphism is a equivalence relation (definition \ref{relations}). 
		\end{theorem}
		\begin{theorem}
		If a mapping $\varphi$ is an isomorphism of $\group$ onto $\group'$
 		\begin{enumerate}
		\item $e\in\group$ is the identity, so that $\varphi(e)=e'\in\group$ is the identity of $\group'$.
		\item $a\in\group$ has inverse $a^{-1}$, so that $\varphi(a^{-1})=(\varphi(a))^{-1}$.
		\end{enumerate}
		\end{theorem}
	\subsection{Transformation Groups}
	\begin{definition}
	For an arbitrary set $S$, let $\transG(S)$ to be semi-group of transformations (mapping) of $S$ into itself. Generally, a transformation group (in $S$) is any subgroup of a group $\transG(S)$ in which the definition \ref{defGroup} holds.
	\end{definition}
	\begin{definition}
	The special case in which $S$ is the set of $n$ numbers, $\transG(S)$ is called symmetric group of degree n, donated $S_n$. For $\a\in S_n$, we write 
	$\nm{
	1 & 2 & \cdots & n\\
	1\a & 2\a & \cdots & n\a\\
	}$
	to represent the mapping order.
	\end{definition}
	\begin{theorem}
	Any group is isomorphic to a transformation group.
	\end{theorem}
	\begin{corollary}
	Any finite group of order $n$ is isomorphic to a sub-group of $S_n$.
	\end{corollary}
	
	$\mathfrak{ABCDEFGHIJKLMNOPRSTUVXWYZ}$
	
	$\mathfrak{abcdefghijklmnopqrstuvwxyz}$
	\section{Rings}
	\section{Fields}
	
\end{document}